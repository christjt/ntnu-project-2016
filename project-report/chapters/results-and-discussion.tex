This chapter shows the results of the simulations done with the framework shown in chapter \ref{ch:methodology} as well as discussing these results.
The chapter is mainly divided into two sections, the first explaining the configuration of the simulation as well as results depicted by graph data.
The second section focuses on discussing the results regarding the reason they have certain values and graphs, as well as comparisons between the different simulations and discovery of correlations and behavioural similarities.

\section{Experimental Results}
As stated in the chapter introduction, this section covers the results of the simulations. 
Even though the simulator supports a wide variety of different configurations, three main simulation groups were chosen for this particular study.
The first group considers different connection port configurations for the robots.
The second simulation group varies the difficulty of the environment.
The third and final simulation group observes the effect of local communication between the robots in a self-assembly structure(the implementation is shown in sec. \ref{sec:local_comm}).
The two first simulation groups targets to contribute to the three main self-assembly mechanisms discussed in chapter \ref{ch:background}(Self-assembly Architectures, Hardware Mechanisms and Assembly Protocols).
The local communication simulation group aims to study the effect of giving the robots the ability to perform simple communication between the robots in a group.

\subsection{Port Configuration}
The first group of simulations that were done was varying the number of connection ports of the robots and their configuration.
These simulations were done such that one could record and discuss the impact of connections between the robots.
In more detail, this group of experiments was conducted such that one could record:

\begin{itemize}
	\item How the number of ports affected the general performance of the system based on the number of ports.
	\item If there is any noticeable difference in the self-assembly architecture.
	\item In what way the port configuration promotes self-assembly, both regarding the sizes of the different robot groups, and the number of groups.
	\item How the port configuration affects the fitness of the experiment, regarding convergence and the final result.
\end{itemize}

The goal of running these simulations is to make correlations between its results to show how configuring the hardware mechanism can affect a self-assembly system.

For this group of simulations, the most relevant statistics will be the difference between group actions versus single robot actions.
The following pages show the recorded data for simulations using, two connection ports, three connection ports, and four connection ports.


\begin{figure}[H]
	\centering
	\begin{subfigure}[b]{0.31\textwidth}
		\centering
		\fbox{\includegraphics[height=\linewidth]{chapters/res/2-conn.png}}
		\caption{2-ports}
	\end{subfigure}
	\begin{subfigure}[b]{0.31\textwidth}
		\centering
		\fbox{\includegraphics[height=\linewidth]{chapters/res/3-conn.png}}
		\caption{3-ports}
	\end{subfigure}
	\begin{subfigure}[b]{0.31\textwidth}
		\centering
		\fbox{\includegraphics[height=\linewidth]{chapters/res/4-conn.png}}
		\caption{4-ports}
	\end{subfigure}
	\caption{The three different port configurations used in simulation}
	\label{fig:collective-behaviour}
\end{figure}


Common configuration parameters of importance for the simulations are listed in table \ref{port-eniornment-config}.

\begin{table}[H]
	\centering
	\begin{tabular}{ @{} l @{\hspace{1cm}}l @{}}
		\toprule 
		Parameter & Value \\ 
		\midrule 
		Number of Robots & 20 \\ 
		Iterations per Generation & 10000 \\
		Scenarios & 3 \\ 
		Generations & 150 \\ 
		\bottomrule 
		
	\end{tabular}
	\caption{The simulation parameters for the environments.}
	\label{port-eniornment-config}
\end{table}



\newpage

\pagestyle{plain}

\vspace*{\fill}

\insertresultgraphs{chapters/generated-graphs/2-ports/fitness-out-2-ports.tex}{chapters/generated-graphs/3-ports/fitness-out-3-ports.tex}{chapters/generated-graphs/4-ports/fitness-out-4-ports.tex}{The \textbf{fitness} results from connection port simulations}{fitness}

Figure \ref{fig:results-ports-fitness} show the results for achieved fitness for the port configuration simulations.
The two and four connection port results are very similar where the four connection port performs slightly better with an average fitness of about 0.2 on generation 1 rising to about 0.5 on generation 150. 
The three connection port simulation performs worse, concerning fitness, in every aspect compared to the other port configurations.

\vspace*{\fill}
\newpage
\vspace*{\fill}

\insertresultgraphs{chapters/generated-graphs/2-ports/group_distribution-out-2-ports.tex}{chapters/generated-graphs/3-ports/group_distribution-out-3-ports.tex}{chapters/generated-graphs/4-ports/group_distribution-out-4-ports.tex}{The \textbf{group distribution} results from connection port simulations}{group-distribution}

Figure \ref{fig:results-ports-group-distribution} represents the distribution of group sizes formed during simulation.
The size of the circles indicates the number of groups formed.
The two and four connection port simulations have more groups of all sizes with the exception of a single group of size five which was formed from the three connection port simulation.
The four connection port simulation formed larger groups than the two connection port simulation.

\vspace*{\fill}
\newpage
\vspace*{\fill}

\insertresultgraphs{chapters/generated-graphs/2-ports/number_of_groups-out-2-ports.tex}{chapters/generated-graphs/3-ports/number_of_groups-out-3-ports.tex}{chapters/generated-graphs/4-ports/number_of_groups-out-4-ports.tex}{The \textbf{number of groups} results from connection port simulations}{number-of-groups}

Figure \ref{fig:results-ports-number-of-groups} shows the number of groups which were formed in the different simulations.
It is seen that the results of the two and four connection port simulations are very similar where the only notable difference is that the four connection port results seem to converge at a faster rate.
The three connection port results are quite poor, having few groups throughout the trial.

\vspace*{\fill}
\newpage
\vspace*{\fill}

\insertresultgraphs{chapters/generated-graphs/2-ports/robots_eaten-out-2-ports.tex}{chapters/generated-graphs/3-ports/robots_eaten-out-3-ports.tex}{chapters/generated-graphs/4-ports/robots_eaten-out-4-ports.tex}{The \textbf{robots eaten} results from connection port simulations}{robots-eaten}

Figure \ref{fig:results-ports-robots-eaten} shows the number of robots which were eaten by predators during simulation.
The four connection port simulation performs best, but only slightly better than the two connection port robots.
The four connection port results converges faster and has slightly better results at the end of the simulation.
The three connection port results are quite poor in comparison where a lot more robots are consumed by predators.

\vspace*{\fill}
\newpage
\vspace*{\fill}


\insertresultgraphs{chapters/generated-graphs/2-ports/robots_starved-out-2-ports.tex}{chapters/generated-graphs/3-ports/robots_starved-out-3-ports.tex}{chapters/generated-graphs/4-ports/robots_starved-out-4-ports.tex}{The \textbf{robots starved} results from connection port simulations}{robots-starved}

Figure \ref{fig:results-ports-robots-starved} shows the number of robots starved each generation.
The results for two and four connection port simulations are very similar where the average number of robots starved is about four at the final generation(150).
The three connection port simulation performs slightly better on these results where the average number of starved robots is slightly less than 3.
The three connection port simulation performs best on this result because most of the robots have been consumed by a predator before they die of starvation.

\vspace*{\fill}
\newpage
\vspace*{\fill}


\insertresultgraphs{chapters/generated-graphs/2-ports/energy_consumed_by_group-out-2-ports.tex}{chapters/generated-graphs/3-ports/energy_consumed_by_group-out-3-ports.tex}{chapters/generated-graphs/4-ports/energy_consumed_by_group-out-4-ports.tex}{The \textbf{energy consumed by group} results from connection port simulations}{energy-consumed-by-group}

Figure \ref{fig:results-ports-energy-consumed-by-group} shows the energy consumed by groups of robots during simulation.
From these results, it can be viewed that the results containing 2 and four connection ports are very similar with an average result of about 60 energy items consumed at generation 150.
The three port configuration performs a lot worse with a result of about ten energy items consumed at generation 150.
The results are correlated with the results obtained from the number of groups formed in the simulation(figure \ref{fig:number-of-groups-3-ports}).


\vspace*{\fill}
\newpage
\vspace*{\fill}


\insertresultgraphs{chapters/generated-graphs/2-ports/energy_consumed_by_robot-out-2-ports.tex}{chapters/generated-graphs/3-ports/energy_consumed_by_robot-out-3-ports.tex}{chapters/generated-graphs/4-ports/energy_consumed_by_robot-out-4-ports.tex}{The \textbf{energy consumed by robot} results from connection port simulations}{energy-consumed-by-robot}

Figure \ref{fig:results-ports-energy-consumed-by-robot} shows the total amount of energy which is consumed by robots which are not self-assembled.
All of the graphs have similar results and slopes, with the exception that the three connection port simulation performs worse.
The reason the average results of figure \ref{fig:energy-consumed-by-robot-2-ports} and \ref{fig:energy-consumed-by-robot-4-ports} flattens out and does not increase after around generation 30 is that more of the robots are self-assembling and hence is not tracked as a part of these results.
As the energy consumed is not decreasing because more robots are a part of groups, it can be deduced that more energy is consumed on a per robot basis.

\vspace*{\fill}
\newpage
\vspace*{\fill}


\insertresultgraphs{chapters/generated-graphs/2-ports/predators_eaten-out-2-ports.tex}{chapters/generated-graphs/3-ports/predators_eaten-out-3-ports.tex}{chapters/generated-graphs/4-ports/predators_eaten-out-4-ports.tex}{The \textbf{predators eaten} results from connection port simulations}{predators-eaten}

Figure \ref{fig:results-ports-predators-eaten} tracks the number of predators that have been eaten by robot groups. 
The four connection port simulation performs best and is correlated with having larger group sizes than the other port configuration shown in figure \ref{fig:group-distribution-4-ports}.
As the robot groups must be of at least size three to consume a predator, the results shown in this figure conform with the other results shown earlier.

\vspace*{\fill}


\newpage

\pagestyle{main}

\subsection{Environment Difficulty}
The motivation for this experiment is to observe how changing the difficulty of the environment affects the evolved behaviour.
For the experiment two environment difficulties were constructed, one easier and one more difficult.
These environment difficulties were then used in the simulations so that the impact of the evolutionary pressure could be recorded and discussed.
More specifically the experiments were constructed to investigate how differing environment difficulties affect the following properties of the evolved behaviour.

\begin{itemize}
	\item The amount of robot groups formed through self-assembly, and the number of robots in each group.
	\item How the energy gathering behaviour is affected. Whether the robots prefer gathering energy individually, or as robot groups. 
	\item How the environment difficulty affects the fitness of the experiment, regarding convergence and the final result.
\end{itemize}


The environment difficulty is modified by varying the following simulation parameters.
The initial energy level the robots, the maximum amount of energy each robot can be charged with, the amount of food in the environment, and the number of predators present in the simulation.
Table \ref{tab:environment-difficulty} presents the simulation parameters that are varied for the environments.

\begin{table}[H]
	\centering
	
	\begin{tabular}{@{} l c c c c @{}}
		\toprule 
		Environment & \small{Predators} & \small{Initial energy} & \small{Food items} & \small{Maximum energy} \\ 
		\midrule 
		\small{Easy environment} & 4 & 8000 & 25 & 10000 \\ 
		\small{Hard environment} & 7 & 6000 & 20 &8000 \\ 
		\bottomrule 
		
	\end{tabular} 
	\caption{The simulation parameters for the environments.}
	\label{tab:environment-difficulty}
\end{table}

The results were obtained by running 20 simulation trials for each difficulty, where each of the trails run for 150 generations.

\newpage
\pagestyle{plain}

\vspace*{\fill}

	\begin{center}
		\subsubsection{Fitness}
		\vspace*{-0.6cm}
	\end{center}

	\insertresultgraphstwo{chapters/generated-graphs/easy/fitness-easy-out.tex}{chapters/generated-graphs/hard/fitness-hard-out.tex}{Figure shows "fitness" results from connection port simulations}{fitness}

These graphs show the results for achieved fitness for the environment difficulty simulations.
The results for the easy environment are noticeably better than the results from the hard environment, starting with an average fitness of 0.24 at generation 1 and rising to 0.55 at generation 150. 

\vspace*{\fill}

\newpage
\vspace*{\fill}
\begin{center}
	\subsubsection{Group Distribution}
	\vspace*{-0.6cm}
	

\end{center}

\insertresultgraphstwo{chapters/generated-graphs/easy/group_distribution-easy-out.tex}{chapters/generated-graphs/hard/group_distribution-hard-out.tex}{Figure shows "group distribution" results from connection port simulations}{group-distribution}

	This graph presents the distribution of group sizes formed during simulation. The distribution for the easy and hard environment are very similar, but one can see that the easy environment simulation have slightly more groups of two and three robots.
	
\vspace*{\fill}
\newpage
\vspace*{\fill}
\begin{center}
	\subsubsection{Number of Groups}
\end{center}

\insertresultgraphstwo{chapters/generated-graphs/easy/number_of_groups-easy-out.tex}{chapters/generated-graphs/hard/number_of_groups-hard-out.tex}{Figure shows "number of groups" results from connection port simulations}{number-of-groups}

	These graphs shows the average number of groups formed at a given timestamp in the simulation.
	One can see that the curves for both simulations are quite similar, but the number of groups formed in the easy environment is around 1.0 more at a given generation.
	\vspace*{-0.6cm}
\vspace*{\fill}
\newpage
\vspace*{\fill}
\begin{center}
	\subsubsection{Robots Eaten}
	\vspace*{-0.6cm}
\end{center}

\insertresultgraphstwo{chapters/generated-graphs/easy/robots_eaten-easy-out.tex}{chapters/generated-graphs/hard/robots_eaten-hard-out.tex}{Figure shows "robots eaten" results from connection port simulations}{robots-eaten}

These graphs show the number of robots which were eaten by predators during the simulation.

\vspace*{\fill}
\newpage
\vspace*{\fill}
\begin{center}
	\subsubsection{Robots Starved}
	\vspace*{-0.6cm}
\end{center}

\insertresultgraphstwo{chapters/generated-graphs/easy/robots_starved-easy-out.tex}{chapters/generated-graphs/hard/robots_starved-hard-out.tex}{Figure shows "robots starved" results from connection port simulations}{robots-starved}
\vspace*{\fill}
\newpage
\vspace*{\fill}
\begin{center}
	\subsubsection{Energy Consumed by Group}
	\vspace*{-0.6cm}
\end{center}

\insertresultgraphstwo{chapters/generated-graphs/easy/energy_consumed_by_group-easy-out.tex}{chapters/generated-graphs/hard/energy_consumed_by_group-hard-out.tex}{Figure shows "energy consumed by group" results from connection port simulations}{energy-consumed-by-group}
\vspace*{\fill}
\newpage
\vspace*{\fill}

\begin{center}
	\subsubsection{Energy Consumed by Robot}
	\vspace*{-0.6cm}
\end{center}

\insertresultgraphstwo{chapters/generated-graphs/easy/energy_consumed_by_robot-easy-out.tex}{chapters/generated-graphs/hard/energy_consumed_by_robot-hard-out.tex}{Figure shows "energy consumed by robot" results from connection port simulations}{energy-consumed-by-robot}
\vspace*{\fill}
\newpage
\vspace*{\fill}
\begin{center}
	\subsubsection{Predators Eaten}
	\vspace*{-0.6cm}
\end{center}

\insertresultgraphstwo{chapters/generated-graphs/easy/predators_eaten-easy-out.tex}{chapters/generated-graphs/hard/predators_eaten-hard-out.tex}{Figure shows "predators eaten" results from connection port simulations}{predators-eaten}
\vspace*{\fill}


\newpage
\pagestyle{main}

\subsection{Local Communication}
\label{sec:local_communication}
The goal of this experiment is to investigate the impact local communication has on the behaviour of the robots.
The experiment is performed by selecting the fittest genomes found during a simulation, removing the communication module, and then observing the change in behaviour, if any.

The observed robot behaviour can be split into two phases, the individual behaviour, and the group behaviour.

\subsubsection{Individual behaviour}
\label{sec:invdividual_behaviour}
The individual robots have two observed movement strategies.
The strategy which is employed depends on if a wall is within sensor its range.
The first strategy involves moving in a wide circular path.
This strategy occurs when no walls are detected by the sensors.
This strategy allows the robots to more collect energy, and will attempt connections with other robots if they collide.
However, they make no attempt to avoid predators in their path.
This behaviour is usually observed at the beginning of the simulation since most of the robots are initialized away from the walls.

The circular motion of the robots is wide enough to make them crash into the walls of the environment.
The behaviour of the robots changes when the sensors detect a wall.
Instead of moving in circles the robot changes its movement pattern to follow the wall of the environment.
Figure \ref{fig:individual-wall-drive} shows how a robot follows the wall while keeping it in sensors range.

\begin{figure}[H]    
	\centering
	\fbox{\includegraphics[width=0.65\textwidth]{chapters/res/wall-hug.png}}
	\caption{A robot using its sensors to follow a wall.}
	\label{fig:individual-wall-drive}
\end{figure}


The robot will continue moving along the wall until it meets another robot, and can form a group, or if the sensors detect a bypassing robot group.
If a robot group comes within sensor range while the robot is moving along the wall, the robot will abandon the wall and attempt to follow the group instead.

\subsubsection{Group behaviour}


\begin{figure}[H]
	
	\centering
	\fbox{\includegraphics[width=0.65\textwidth]{chapters/res/group_circles.png}}
	\caption{Robot groups moving in circles.}
	\label{fig:group-circles}
\end{figure}

The behaviour of the robot groups is similar to the first mode of the individual robots.
Robot groups also move in wide circles, displayed in figure \ref{fig:group-circles}, but if the group crashes into a wall, it will simply turn around and continue.
The robot groups consume predators in their path, but they make no attempt to follow detected predators.
The robot groups will continue to drive in circles, consuming energy, predators, and making connections, until the end of the simulation, or until the members of the groups starve.

\subsubsection{Local communication}
\label{sec:disable-local-communication}
Disabling the local communication module has a significant effect on the behaviour.
With the communication disabled the robots will no longer switch to the group behaviour once they are connected.
Instead, the groups will continue to perform the individual robot behaviour regardless of the number of connected robots.

\section{Discussion}
This section covers the analysis of the obtained results.
The section is split into three parts.
The first part covers the connection port simulations and reviews the impact these results have on achieving self-assembly.
The second part covers the different environmental difficulty simulations and.
The third part covers the local communication module and explains the effect it has on the robot system.

\subsection{Port configuration Analysis}
Regarding the results fetched from the port configuration simulations the first obvious remarks stem from the simulation running a three port configuration.
The results of this simulation are much weaker concerning performance and promotion of self-assembly than the simulations running two and four connection ports.
The reason for this can not be deduced completely from the empirical results, but as the only difference in these simulations are the number of connection ports and the alignment; it is clear that the connection port configuration can significantly impact the performance of the simulation.
It can also be deduced that it is not the number of connection ports that has the primary impact of the solution, but rather the placement.
The reason one can make this claim is that the configuration using two connection ports and four connection ports perform quite similar in terms of performance.
If the number of connection ports had a significant impact on the results, one would expect the simulation using either two or four connection ports to yield even poorer results than the three connection port simulation.
This effect narrows the port configuration problem down to the alignment of the connection ports.

\begin{figure}[H]
	\centering
	\begin{subfigure}[b]{0.31\textwidth}
		\centering
		\fbox{\includegraphics[height=\linewidth]{chapters/res/2-ports-robot.png}}
		\caption{2-ports}
	\end{subfigure}
	\begin{subfigure}[b]{0.31\textwidth}
		\centering
		\fbox{\includegraphics[height=\linewidth]{chapters/res/3-ports-robot.png}}
		\caption{3-ports}
	\end{subfigure}
	\begin{subfigure}[b]{0.31\textwidth}
		\centering
		\fbox{\includegraphics[height=\linewidth]{chapters/res/4-ports-robot.png}}
		\caption{4-ports}
	\end{subfigure}
	\caption{The three different port configurations used in simulation}
	\label{fig:robot-port-configuration}
\end{figure}

Figure \ref{fig:robot-port-configuration} shows a closer view of the alignment that the robots initially have when spawned into the environment.
As explained in \ref{sec:modifications}, the robots can rotate their connection ports as a group.
A standard strategy which is usually evolved is to either constantly rotate the connection ports in hopes of lining up the ports to another robot, or, the robots start rotating their ports when the sensors see another robot in an effort to self-assemble.
There are two main problems that the three connection port robots have compared to the other port configurations.
First, the initial port location does not align to any other robot.


\begin{figure}[H]
	\begin{subfigure}[t]{0.49\textwidth}
		\centering
		\fbox{\includegraphics[height=0.9\linewidth]{chapters/res/3-ports-alignment.png}}
		\caption{Initial alignment}
		\label{3-port-guided-allignment}
	\end{subfigure}
	\begin{subfigure}[t]{0.49\textwidth}
		\centering
		\fbox{\includegraphics[height=0.9\linewidth]{chapters/res/3-ports-alignment-offset.png}}
		\caption{After $50^{\circ}$ port rotation.}
		\label{3-port-guided-allignment-offset}
	\end{subfigure}
	\caption{This figure shows how the 3 connection port robots align}
\end{figure}


As seen in figure \ref{3-port-guided-allignment}, there is not a trivial alignment for the robots to connect.
One might initially presume that this is not a problem as the robots have a mechanism for rotating their ports to solve this exact issue.
However, as all the robots are running the same genome, as per off-line evolution and hence the same behaviour, it becomes increasingly difficult for the robots to solve this problem.
As explained earlier, the robots in this simulation tend to evolve a strategy which involves constantly spinning the connection ports in one direction.
However, as seen in figure \ref{3-port-guided-allignment-offset}, in the case where all robots at some time step have rotated their connection ports $50^{\circ}$, the same issue of port alignment would still hold.

\begin{figure}[H]
	\begin{subfigure}[t]{0.49\textwidth}
		\centering
		\fbox{\includegraphics[height=0.9\linewidth]{chapters/res/2-ports-alignment.png}}
		\caption{2 connection ports alignment}
		\label{2-port-guided-allignment}
	\end{subfigure}
	\begin{subfigure}[t]{0.49\textwidth}
		\centering
		\fbox{\includegraphics[height=0.9\linewidth]{chapters/res/4-ports-alignment.png}}
		\caption{4 connection ports alignment}
		\label{4-port-guided-allignment}
	\end{subfigure}
	\caption{This figure shows how the 2 and 4 connection ports robots align from initial configuration}
	\label{2-4-port-guided-allignment}
\end{figure}

Consider figure \ref{2-4-port-guided-allignment}.
In this example, there are two and four port configurations.
It can be observed that with an initial rotation of the ports, there exist possibilities for the robots to self-assemble without having the robots behave differently in terms of rotating their connection ports.
This differentiation seems to be the main reason that the three connection port configuration is being outperformed.

The second problem robots with three connection ports, in this alignment, is the possible group formations the robots can form.
Chapter \ref{ch:background} covers the chain and lattice architectures that the robots can form when self-assembling. %needs ref/explain
The simulator is developed to support the lattice architecture because if its simplistic method of coordinated movement.

\begin{figure}[H]
	\begin{subfigure}[t]{0.49\textwidth}
		\centering
		\fbox{\includegraphics[height=0.9\linewidth]{chapters/res/2-4-port-architectures.png}}
		\caption{Robot groups with 2 and 4 connection ports}
		\label{2-4-port-architecture}
	\end{subfigure}
	\begin{subfigure}[t]{0.49\textwidth}
		\centering
		\fbox{\includegraphics[height=0.9\linewidth]{chapters/res/3-ports-architecture.png}}
		\caption{Robot groups with 3 connection ports}
		\label{3-port-architecture}
	\end{subfigure}
	\caption{This figure contains self-assembled robot groups with different assembly combinations}
	\label{port-architectures}
\end{figure}

It can be seen from figure \ref{port-architectures} that the different connection port configurations create various types of groups. 
With two and four connection ports (figure \ref{2-4-port-architecture}), the robot groups either take the form of a line or some square grid formation.
Possible formations of the three connection ports robot groups(figure \ref{3-port-architecture}) brakes the pattern of a square grid configuration which makes it harder for other robots trying to connect to the group.
The main reason for this connection problem is the relative position a connecting robot needs, is harder to attain because of the larger distance between the connection ports.

There are not significant discrepancies between the results from the port configuration simulation containing two and four connection ports.
The only result which differs significantly is "predators eaten"(figure \ref{fig:predators-eaten-2-ports} and \ref{fig:predators-eaten-4-ports}).
The reason for this can be deduced from figure \ref{fig:group-distribution-2-ports} and \ref{fig:group-distribution-4-ports} which shows that robots with 4 connection ports tend to form larger groups.
It can however be viewed from figure \ref{fig:number-of-groups-2-ports} and \ref{fig:number-of-groups-4-ports} that 2 and 4 connection ports have roughly the same number of groups.
The occurrence of a greater amount of larger groups naturally agrees with eaten more predators as groups need to be of at least size three to consume a predator.
The reason for four connection ports robots to attain larger groups is simply that more connection ports allow more points of entry for other robots trying to connect, which increases the probability of succeeding self-assembly to the group.

From these results, it can be deduced that larger groups do not give rise to better fitness in this experiment, but rather the number of groups (a group is of minimum size 2) correlates with the fitness.
The reason for this is that the robots in the port configuration simulations are not in a great need of energy.
The robots are able to naturally attain what they need in the environment and hence do not have to rely on a strategy involving predator consumption.

\subsection{Environmental difficulty analysis}
The analysis and discussion on the impact of environmental difficulty can be divided into three main categories.
The impact of environmental threats when the difficulty is modified.
How the promotion of self-assembly is affected by the environmental difficulty.
How the evolved energy collection strategy is affected by the environment difficulty.	

\subsubsection{Environmental threats}
The analysis and discussion on the impact of environmental difficulty can be divided into three main categories.
The impact of environmental threats when the difficulty is modified.
How the promotion of self-assembly is affected by the environmental difficulty.
How the evolved energy collection strategy is influenced by the environment difficulty.    

\subsubsection{Environmental threats}
The robots have two threats in the environments presented, starvation and getting killed by predators.
The robots in the easy environment receive more energy from food, and there is more food available.
From figure \ref{fig:results-env-robots-starved} one can see that this reduces the amount of robots dying from starvation in the easy environment, but the improvement is minuscule.

Increasing the number of predators in the environment seems to have a higher impact on the difficulty presented by an environment.
Figure \ref{fig:results-env-robots-eaten} shows that increasing the amount of predators present in the environment has a greater impact on the difficulty of the environment than limiting the energy available.
The reason for why increasing the number of predators has a much higher impact on difficulty is not completely clear from the results.
However, the observed behaviour described in section \ref{sec:invdividual_behaviour} can help explain the results.
The sensors are used by the robot to detect walls and other robots, but not predators or food.
This behaviour means that the robot may miss some food, but there is enough food in the environment so the robot will eventually find more food.
On the other hand, failing at predator avoidance has much more severe consequences as the predator will kill the robot.

\subsubsection{Promotion of self-assembly}
One of the motivations for this experiment was to see how modifying the evolutionary pressure affects the promotion of self-assembly.
Figure \ref{fig:results-env-number-of-groups} show that the robots form more groups in the easy environment.
At first glance, this seems to indicate that the easy environment is more successful at promoting self-assembly.
This difference in the number of groups may be explained by examining the lifetime of the robots.
As explained in section \ref{sec:evaluation}, the fitness of a genome is determined by the average lifetime of a robot.
The fitness achieved in the easy environment, figure \ref{fig:fitness-easy}, is higher than the fitness achieved in the hard environment, figure \ref{fig:fitness-hard}.
The fitness means that the robots in the easy environment live longer, and as a consequence have more time to form groups.

However, figure \ref{fig:results-env-group-distribution} shows that the size of the robot groups formed is not affected by modifying the difficulty of the environment.
In both environments, the distribution of group sizes is heavily weighted towards groups of two.
The reason for this may be that the environments give a high reward for being in a group.
That is protection from predators, but there is no additional reward for forming larger groups.

\subsubsection{Energy collection strategy}
One can see from the figures \ref{fig:results-env-energy-consumed-by-robot} and \ref{fig:results-env-energy-consumed-by-group} that the robots in the easy environment collect far more energy than the robots in the difficult environment as individual robots and robot groups.
This result can likely also be attributed to the fact that the robots in the easy environment live longer, and that there is more energy available, instead of a more optimal energy gathering strategy.
The reasoning for this is as follows.
One can look at the ratio of energy collected by individual robots versus energy collected by groups of robots for the environments.
This relationship is presented in table \ref{tab:energy-collected-ratio}.
\begin{table}[H]
	\centering
	\begin{tabular}{ l @{\hspace{1cm}}c @{\hspace{1cm}}c }
		\toprule
		Generation & Easy & Hard \\ 
		\midrule 
		10     & 54\%    & 51\%  \\ 
		50     & 71\%    & 67\% \\ 
		100     & 71\%    & 71\% \\ 
		150     & 72\%     & 73\% \\ 
		\bottomrule 
	\end{tabular} 
	\caption{The percentage of energy collected by groups of robots for the environments.}
	\label{tab:energy-collected-ratio}
\end{table}

Table \ref{tab:energy-collected-ratio} shows that in both environments the ratio of energy collected by robot groups is approximately the same.
The ratio means that although the robots in the easy environment collect more energy in total, the strategies evolved in the different environments are similar.
This also coincides with that the observed behaviour described in section \ref{sec:local_communication} is very similar for the different environments. \ref{sec:local_communication} is very similar for the different environments.

\subsection{Local communication analysis}
As described in section \ref{sec:disable-local-communication}, the evolved behaviour makes use of the communication module.
When the communication module was disabled, the robots did not change their behaviour when they formed groups.
Therefore, it is reasonable to assume that local communication is at least involved in modifying the robot behaviour once connected to a group.
Exactly how the evolved neural network interprets the messages received is challenging to infer, but one can observe the messages sent to get a guiding concept.

\begin{figure}[H]
	\centering
	[0.992 0.999 0.423 0.002]
	
	\caption{The message passed between the robots.}    
	\label{fig:message_default}
\end{figure}

Without any other inputs, all robots send the message displayed in \ref{fig:message_default} by default.
Receiving other inputs, such as sensors, changes the message by a negligible amount.
The surprising thing about the message is that the components in the communication messages have wildly different values.
It turns out that the values in the messages have an interesting interaction with the port connection status that is also propagated to the neural network.
The port connection status contains the connection status of each port the robot has.

\begin{table}[H]    
	
	\begin{tabular}{c c}
		\toprule
		Message:[0.992 0.999 0.423 0.002] & Message: [1.0 1.0 1.0 1.0] \\
		\toprule
		\begin{tabular}[t]{ @{} c c @{}}
			\toprule
			\small{Port status} & \small{Desired rotation\textsubscript{deg/step}} \\ 
			\midrule 
			1 1 0 0  &    0.9482  \\ 
			1 0 1 0 & 0.999  \\ 
			1 0 0 1 & 0.517  \\ 
			0 1 1 0 & 0.997  \\ 
			0 1 0 1 & 0.705  \\ 
			0 0 1 1 & 0.997  \\ 
			\bottomrule 
		\end{tabular} 
		&
		\begin{tabular}[t]{@{} c c @{}}
			
			\toprule 
			\small{Port status} & \small{Desired rotation\textsubscript{deg/step}} \\ 
			\midrule 
			1 1 0 0 & 0.719  \\ 
			1 0 1 0 & 0.976  \\ 
			1 0 0 1 & 0.658  \\ 
			0 1 1 0 & 0.866  \\ 
			0 1 0 1 & 0.674  \\ 
			0 0 1 1 & 0.822  \\ 
			\bottomrule     
		\end{tabular}     
	\end{tabular}
	\caption{The resulting desired rotations for different port combinations with the evolved message and a test message for comparison. A port status value of 1 means the particular port is connected, 0 means it is not connected.}
	\label{tab:port-desired-rotation}
\end{table}

Table \ref{tab:port-desired-rotation} shows how the desired rotation for the robots varies with the local topology of the connected robots.
The table shows this variation with the evolved message and a dummy message for comparison.
It can be observed that the resulting desired rotations for the robots have different values for the two messages.
The desired rotation for the robots determines the radius of the circles that dictates the robots' movements.
These results show that the local communication module is used for two purposes.
The first purpose is to act as a switch to change from the individual robot behaviour to the group behaviour.
Additionally, the communicated message decides the robot group's behaviour depending on the different connection topologies.




























