The main focus and goal of this thesis has been to discover and test mechanisms of self-assembly.
The experiment has been conducted using a heavily modified version of the roborobo framework.
The main factors that have been researched in accordance with self-assembly have been: connection port configuration, environmental influence and local communication.

The results obtained from the connection port simulations, show that configuration of the connection ports can greatly impact the emergence of self-assembly using an evolutionary algorithm.
The port configuration consists of the number of connection ports each robot has available and the relative positioning of the connection ports on the robot.
Both elements influence the size and frequency of self-assembling robot groups.
It can however be narrowed down to a single influential self-assembly mechanism; the assembly protocol.
It is clear from the results that providing the robots with tools that allow an efficient and simple assembly protocol to be evolved, is essential to achieve good results.

The main focus of the learning algorithm should be to solve the task at hand and not deriving a complex strategy for achieving self-assembly.
Having the ability to form a complex assembly protocol can be appropriate in a particular situation as it may greatly improve performance.
However, it should not be a minimum requirement for the ability to self-assemble.
An evolutionary algorithm performs better when an incremental solution to some desired behaviour is possible.
If the least complex achievable self-assembly protocol requires sufficiently advanced cognition, then the robots may only have a few occurrences of self-assembly or none at all.

The results gained from the environment difficulty simulations implies that the difficulty of the environment is not directly correlated with promoting self-assembly.
The results show that the robots perform better in an easy environment, but this is rather due to restrictions in the environment and not due to the ability to self-assemble.
The only effect that difficult environments impose on self-assembly is making robots die earlier, giving them less opportunities to self-assemble.


\section{Future work}
The evolved strategies and the observed behaviour for the easy and the difficult environment were very similar.
It would be of interest to continue modifying the environments to find the threshold where new strategies have to be evolved. 