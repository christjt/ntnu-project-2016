Every year the expansion and development of robots increases. 
We see robots being used in large military operations in the form of drones, as well as small robots in households that can vacuum your house while you are gone.
A lot of this technology has only recently been developed and it is expected that it will continue to grow and become a part of almost every aspect of society.

As a part of advancing robotics, a lot of the different functionality and behaviour of robots are being explored.
These mechanisms include self-replicating machines, artificial intelligence, self-assembly, inter-robot communication, etc.
In this study we will be exploring a robot's ability to self-assemble. 

Self-assembly is the autonomous organization of components into patterns or structures without human intervention \cite{whitesides_self-assembly_2002}.
We can observe self-assembling processes in nature at a molecular scale \cite{heylighen_science_2001} as well as at a planetary scale in the form of solar systems; each of which presents many different configurations.
There are several reasons to study self-assembly.
One reason is, the cells in the human body self-assemble, so understanding this mechanism can help us understand life and the prerequisites for it.
In chemistry it has been discovered that polymers can self-assemble from monomers in order to form materials that can be used in a wide range of applications\cite{chung_use_2004}\cite{siracusa_biodegradable_2008}. 
In electronics, the self-assembly of nano particles can be used to create certain semiconductors such as solar panels\cite{henini_chapter_2008}.
In either case, understanding and controlling self-assembly is essential.

Self-assembly is also used as a practical strategy for making nano structures as well as providing beneficial characteristics to the field of robotics.
In a more general sense, self-assembly can improve the movement of robots by being able to overcome larger obstacles in the environment and having a large sensory field by grouping the sensors in a self-assembled structure.

According to \cite{yim_modular_2007}, there are three main characteristics that self-reconfigurable systems benefit from.
The first is versatility, systems can self-assemble into new structures based on the specific task it is to solve.
For example, the system may change from a rolling robot to a snake robot depending on the situation.
The second characteristic is robustness.
Since the robots are usually homogeneous, one can replace broken robots to repair the system.
It is also possible to automate this process, leading to self-repair which makes the system more robust than other traditional systems.
The last characteristic is the potential low cost of developing a self-reconfigurable system.
Constructing many of the same type of robot generally lowers the overall robot cost.
In addition, there are many complex machines that can be built from  a reconfigurable system that saves cost through reuse and generality.
There are clear benefits of using self-assembly mechanisms in robotics.
However, there have been many different approaches to try to achieve self-assembly in different systems of robots.

Self-assembly can form many different configurations based on the topology of the organized structure.
In addition, self-assembly systems vary in their ability to reconfigure and disassemble depending on the challenges the environment provides.
One class of self assembly systems is stochastic reconfiguration. 
With stochastic reconfiguration the robots do not have movement capabilities themselves. 
The robots are moved passively by the environment, and bind to each other on random collisions\cite{gro_autonomous_2006}. 
With this form of reconfiguration, the reconfiguration time can only be guaranteed statistically\cite{yim_modular_2007}.
\cite{bishop_programmable_2005} demonstrates how robots named "programmable parts" can self-organize through passive movements provided by the environment. Systems that use stochastic reconfiguration are most common at micro level environments, and only achieve self-assembly because of the influence by their environment.
These characteristics do not conform to what we are trying to study, so this study will not delve further into this topic.

A crucial part of this study is to observe how self-assembly can transpire as an emergent behaviour through machine learning.
A robot that can improve its performance based on past observations is said to use some form of machine learning.
The main competing algorithms for machine learning in robot systems, as of writing, are reinforcement leaning and evolutionary algorithms.
This study will look at machine learning and self-assembly from an evolutionary algorithms point of view.

This study covers how to implement machine learning in a system that takes advantage of self-assembly.
In particular, how self-assembly can be used in conjunction with evolutionary robotics.
Evolutionary robotics represents a way to automate the design of control systems for autonomous robots, using algorithms based on Darwinian evolution \cite{trianni_evolving_2004}.
Evolutionary robotics is a field of research that lies under the field of artificial intelligence. 
The main purpose of evolutionary robotics is to develop autonomous robot controllers that solves a given task which is not directly programmed by a human.
Evolutionary robotics takes advantage of selective reproduction based on how well the robots solve a certain task.
The controller of a robot is most often represented with an \emph{artificial neural network} where the parameters of the neural network are set by an evolutionary algorithm.

There have been previous studies about using machine learning to promote self-assembly\cite{trianni_evolving_2004}\cite{montanier_adaptive_2014}\cite{li_co-evolution_2015}. 
However, many of these studies uses differing self-assembly mechanisms such as the architecture used by the self-assembly system, the actions that are performed by the robots to achieve an organized structure and the docking mechanism hardware.
Therefore, the motivation of this research is to analyse if any of these mechanisms are superior in terms of promoting self-assembly in regards to learned robot controllers.

The experiment will be done in simulation with the Roborobo platform. 
Roborobo is a light-weight multi-platform simulator for extensive robotics experiments, based on basic robotic hardware setup.
The experiment is a simple predator/pray scenario, where the evolved robots(pray) can self-assemble to gain certain advantages over its pray.
While creating the simulation, great care has been taken into making most of the parameters of the system configurable.
This is done so that testing the different mechanisms is simple, such that we can compare how introducing different mechanisms effects the robot's ability to self-assemble and the behaviour of the robots.

\section{Thesis objectives}
The main focus of this study is to compare the different mechanisms that influences self-assembling robots that are governed by machine learning. 
Relevant mechanisms are the self-assembly architecture, the assembly protocol and docking(assembly) mechanisms. 
The main objectives of this study includes:

\begin{itemize}

\item Implement and explain a simulation that uses various self-assembly mechanisms such as a self-assembly architecture and an assembly protocol.

\item Compare and analyse the results of the simulation. 

\item Based on the analysis, make a conclusion as to the advantage of using certain implementations of self-assembly mechanisms.

\end{itemize}

\section{Problem statement}
The scope of this study covers the self-assembly mechanisms in a system of robots. 
The grand question this study is trying to contribute to is: \textbf{How does the elements present in a system of robots influence the emergence of self-assembly where the robots are given basic learning capabilities?}
To get a better understanding of this field, this study will present methods and analysis to answer the following questions:

\begin{itemize}

\item In what way does an evolved assembly protocol influence robot’s ability to achieve self-assembly?

\item How does the number of connectors provided by the docking mechanism influence the systems ability to self-assemble?

\item How do self-assembly systems scale when using many robots?

\item In what way can the learning algorithm and its implementation influence the self-assembling functionality?

\end{itemize}

In addition to these comparisons of the mechanisms in self-assembly systems, the thesis will also cover collective behaviour in the systems and how this affects the emergence of self-assembly:

\begin{itemize}

\item How does introducing local communication between the robots influence the system's ability to self-assemble and how does it effect the behaviour of self-assembled robots?

\end{itemize}

\section{Structure of report}

