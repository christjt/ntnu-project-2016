\vspace*{\fill}
\insertresultgraphstwo{chapters/generated-graphs/easy/fitness-easy-out.tex}{chapters/generated-graphs/hard/fitness-hard-out.tex}{The \textbf{fitness} from environment simulations}{fitness}

Figure \ref{fig:results-env-fitness} shows the results for achieved fitness for the environment difficulty simulations.
The results for the easy environment are noticeably better than the results from the hard environment, starting with an average fitness of 0.24 at generation 1 and rising to 0.55 at generation 150. 

\vspace*{\fill}

\newpage

\vspace*{\fill}
\insertresultgraphstwo{chapters/generated-graphs/easy/group_distribution-easy-out.tex}{chapters/generated-graphs/hard/group_distribution-hard-out.tex}{The \textbf{group distribution} from environment simulations}{group-distribution}

Figure \ref{fig:results-env-group-distribution} presents the distribution of group sizes formed during simulation. The distribution for the easy and hard environments are very similar, but one can see that the easy environment simulation has slightly more groups of two and three robots.

\vspace*{\fill}
\newpage

\vspace*{\fill}
\insertresultgraphstwo{chapters/generated-graphs/easy/number_of_groups-easy-out.tex}{chapters/generated-graphs/hard/number_of_groups-hard-out.tex}{\textbf{Number of groups} from environment simulations}{number-of-groups}

Figure \ref{fig:results-env-number-of-groups} shows the average number of groups formed at a given timestamp in the simulation.
One can see that the curves for both simulations are quite similar, but the number of groups formed in the easy environment is around one more at any given generation.

\vspace*{\fill}
\newpage
\vspace*{\fill}

\insertresultgraphstwo{chapters/generated-graphs/easy/robots_eaten-easy-out.tex}{chapters/generated-graphs/hard/robots_eaten-hard-out.tex}{\textbf{Number of robots eaten} from environment simulations}{robots-eaten}

Figure \ref{fig:results-env-robots-eaten} shows the number of robots which were eaten by predators during the simulations.
The easy environment simulation performs a bit better than the hard simulation.
In the easy environment simulation, an average of around 16 robots are eaten in the first generation and decreases to around eight robots in generation 150.
In the hard environment simulation, an average of around 17 robots are eaten in the first generation and decreases to around 11 robots in generation 150.

\vspace*{\fill}
\newpage
\vspace*{\fill}

\insertresultgraphstwo{chapters/generated-graphs/easy/robots_starved-easy-out.tex}{chapters/generated-graphs/hard/robots_starved-hard-out.tex}{\textbf{Number of robots starved} results from environment simulations}{robots-starved}

Figure \ref{fig:results-env-robots-starved} shows the number of robots dying from starvation.
The graphs are relatively similar, with the worst case being almost the same.
In the easy environment, compared to the hard environment, there is around one robot less dying from starvation for the average and best case.
Surprisingly, these graphs show that during the first 30 generations the results are getting worse.
The increase in robots starving can be explained by that many robots are getting eaten by predators before they have time to starve in the early generations.  

\vspace*{\fill}
\newpage
\vspace*{\fill}

\insertresultgraphstwo{chapters/generated-graphs/easy/energy_consumed_by_group-easy-out.tex}{chapters/generated-graphs/hard/energy_consumed_by_group-hard-out.tex}{\textbf{Energy consumed by group} from environment simulations}{energy-consumed-by-group}

Figure \ref{fig:results-env-energy-consumed-by-group} shows the total amount of energy consumed by robot groups during simulation.
In the easy environment simulation, the robot groups gather far more energy than in the hard environment simulation.
Gathering an average of 56 pieces of energy at generation 150 for the easy environment compared to just 31 pieces in the hard environment simulation.

\vspace*{\fill}
\newpage
\vspace*{\fill}

\insertresultgraphstwo{chapters/generated-graphs/easy/energy_consumed_by_robot-easy-out.tex}{chapters/generated-graphs/hard/energy_consumed_by_robot-hard-out.tex}{\textbf{Energy consumed by robot} from environment simulations}{energy-consumed-by-robot}

Figure \ref{fig:results-env-energy-consumed-by-robot} shows the total amount of energy eaten by individual robots during simulation.
The easy environment performs better, collecting an average of 14 energy in the first generation and an average of 24 in the final generation.
The amount of energy gathered in both environments flattens out at around 30 generations.
The reason for this is that more robots are self-assembling and is hence not tracked as part of these results.

\vspace*{\fill}
\newpage
\vspace*{\fill}

\insertresultgraphstwo{chapters/generated-graphs/easy/predators_eaten-easy-out.tex}{chapters/generated-graphs/hard/predators_eaten-hard-out.tex}{\textbf{Number of Predators eaten} from environment simulations}{predators-eaten}

Figure \ref{fig:results-env-predators-eaten} shows the number of predators eaten by robot groups.
On average, the amount of consumed predators is about the same for both environment difficulties, with the robots in the hard environment being consumed at a slightly higher rate.
However, in the best case, the hard environment performs significantly better. 
The results may be correlated with it being more predators present in the hard environment that the robots can eat.
\vspace*{\fill}
