\vspace*{\fill}
\begin{center}
	\subsubsection{Fitness}
	\vspace*{-0.6cm}
\end{center}

\insertresultgraphs{chapters/generated-graphs/2-ports/fitness-out-2-ports.tex}{chapters/generated-graphs/3-ports/fitness-out-3-ports.tex}{chapters/generated-graphs/4-ports/fitness-out-4-ports.tex}{Figure shows "fitness" results from connection port simulations}{fitness}

These graphs show the results for achieved fitness for the port configuration simulations.
The 2 and 4 connection port results are very similar where the 4 connection port performs slightly better with an average fitness of about 0.2 on generation 1 rising to about 0.5 on generation 150. 
The 3 connection port simulation performs worse, in terms of fitness, in every aspect compared to the other port configurations.

\vspace*{\fill}
\newpage
\vspace*{\fill}
\begin{center}
	\subsubsection{Group Distribution}
	\vspace*{-0.6cm}
\end{center}

\insertresultgraphs{chapters/generated-graphs/2-ports/group_distribution-out-2-ports.tex}{chapters/generated-graphs/3-ports/group_distribution-out-3-ports.tex}{chapters/generated-graphs/4-ports/group_distribution-out-4-ports.tex}{Figure shows "group distribution" results from connection port simulations}{group-distribution}

This graph represents the distribution of group sizes formed during simulation.
The size of the circles indicate the number of groups formed.
The 2 and 4 connection port simulations have more more groups of all sizes with the exception of a single group of size 5 which was formed from the 3-port simulation.
The 4 connection port simulation formed larger groups than the 2 connection port simulation.

\vspace*{\fill}
\newpage
\vspace*{\fill}
\begin{center}
	\subsubsection{Number of Groups}
	\vspace*{-0.6cm}
\end{center}

\insertresultgraphs{chapters/generated-graphs/2-ports/number_of_groups-out-2-ports.tex}{chapters/generated-graphs/3-ports/number_of_groups-out-3-ports.tex}{chapters/generated-graphs/4-ports/number_of_groups-out-4-ports.tex}{Figure shows "number of groups" results from connection port simulations}{number-of-groups}

These graphs show the number of groups which were formed in the different simulations.
It is seen that the results for the 2 and 4 connection port simulations are very similar where the only notable difference is that the 4 connection port results seem to converge at a faster rate.
The 3 connection port results are quite poor, having few groups throughout the trial.

\vspace*{\fill}
\newpage
\vspace*{\fill}
\begin{center}
	\subsubsection{Robots Eaten}
	\vspace*{-0.6cm}
\end{center}

\insertresultgraphs{chapters/generated-graphs/2-ports/robots_eaten-out-2-ports.tex}{chapters/generated-graphs/3-ports/robots_eaten-out-3-ports.tex}{chapters/generated-graphs/4-ports/robots_eaten-out-4-ports.tex}{Figure shows "robots eaten" results from connection port simulations}{robots-eaten}

These graph show the number of robots which were eaten by predators during simulation.
The 4 connection port simulation performs best, but only slightly better than the 2 connection port robots.
The 4 connection port results converges faster and has slightly better results at the end of simulation.
The 3 connection port results are quite poor in comparison where a lot more robots are consumed by predators.

\vspace*{\fill}
\newpage
\vspace*{\fill}
\begin{center}
	\subsubsection{Robots Starved}
	\vspace*{-0.6cm}
\end{center}

\insertresultgraphs{chapters/generated-graphs/2-ports/robots_starved-out-2-ports.tex}{chapters/generated-graphs/3-ports/robots_starved-out-3-ports.tex}{chapters/generated-graphs/4-ports/robots_starved-out-4-ports.tex}{Figure shows "robots starved" results from connection port simulations}{robots-starved}

These results show the number of robots starved each generation.
The results for 2 and 4 connection port simulations are very similar where the average number of robots starved is about 4 at the final generation(150).
The 3 connection port simulation performs slightly better on this results where the average number of starved robots is slightly less than 3.
The reason for the 3 connection port simulation for perform best on this result is that most of the robots in have been consumed by a predator before they die of starvation.

\vspace*{\fill}
\newpage
\vspace*{\fill}
\begin{center}
	\subsubsection{Energy Consumed by Group}
	\vspace*{-0.6cm}
\end{center}

\insertresultgraphs{chapters/generated-graphs/2-ports/energy_consumed_by_group-out-2-ports.tex}{chapters/generated-graphs/3-ports/energy_consumed_by_group-out-3-ports.tex}{chapters/generated-graphs/4-ports/energy_consumed_by_group-out-4-ports.tex}{Figure shows "energy consumed by group" results from connection port simulations}{energy-consumed-by-group}

These graphs show the energy consumed by groups of robots during simulation.
From these results it can be viewed that the results containing 2 and 4 connection ports are very similar with an average result of about 60 energy packets consumed at generation 150.
The 3 port configuration performs a lot worse with an end result of about 10 energy packets consumed at generation 150.
This is correlated to the results obtained from number of groups formed in the simulation(figure \ref{fig:number-of-groups-3-ports}).


\vspace*{\fill}
\newpage
\vspace*{\fill}

\begin{center}
	\subsubsection{Energy Consumed by Robot}
	\vspace*{-0.6cm}
\end{center}

\insertresultgraphs{chapters/generated-graphs/2-ports/energy_consumed_by_robot-out-2-ports.tex}{chapters/generated-graphs/3-ports/energy_consumed_by_robot-out-3-ports.tex}{chapters/generated-graphs/4-ports/energy_consumed_by_robot-out-4-ports.tex}{Figure shows "energy consumed by robot" results from connection port simulations}{energy-consumed-by-robot}

This graph shows the total amount of energy which is consumed by robots which are not self-assembled.
All of the graphs have similar results and slopes, with the exception that the 3 connection port simulation performs worse.
The reason the average results of figure \ref{fig:energy-consumed-by-robot-2-ports} and \ref{fig:energy-consumed-by-robot-4-ports} flattens out and does not increase after around generation 30 is that more of the robots are self-assembling and hence is not tracked as a part of these results.
As the energy consumed is not decreasing because more robots are a part of groups, it can be deduced that there more energy is consumed on a per robot basis.

\vspace*{\fill}
\newpage
\vspace*{\fill}
\begin{center}
	\subsubsection{Predators Eaten}
	\vspace*{-0.6cm}
\end{center}

\insertresultgraphs{chapters/generated-graphs/2-ports/predators_eaten-out-2-ports.tex}{chapters/generated-graphs/3-ports/predators_eaten-out-3-ports.tex}{chapters/generated-graphs/4-ports/predators_eaten-out-4-ports.tex}{Figure shows "predators eaten" results from connection port simulations}{predators-eaten}

These results track the number of predators that have been eaten by robot groups. 
The 4 connection port simulation performs best which is correlated with having larger group sizes than the other port configuration shown in figure \ref{fig:group-distribution-4-ports}.
As the robot groups must be of at least size 3 to consume a predator, the results shown in this figure conform the other results shown earlier.

\vspace*{\fill}
